\documentclass[letterpaper,pdftex]{article}

\setlength{\textwidth}{168mm}
\setlength{\textheight}{210mm}
\setlength{\oddsidemargin}{0cm}
\setlength{\topmargin}{0cm}
\setlength{\headheight}{48pt}
\addtolength{\textheight}{-25pt}
\voffset -0.5in

\usepackage{natbib}
\usepackage[utf8]{inputenc}
\usepackage[spanish]{babel}
\usepackage{xcolor,graphicx}
\usepackage{fancyhdr}
\usepackage{multirow}
\usepackage{siunitx}
\usepackage{hyperref}
\hypersetup{
    colorlinks,
    citecolor=blue,
    filecolor=black,
    linkcolor=blue,
    urlcolor=black
}
\usepackage{epstopdf}
\usepackage[autolinebreaks,useliterate]{mcode}
\pagestyle{fancy}
\renewcommand{\headrule}{\color{gray}
\hrule width\headwidth height\headrulewidth \vskip-\headrulewidth}
\renewcommand{\footrule}{{\color{gray}
\vskip-\footruleskip\vskip-\footrulewidth
\hrule width\headwidth height\footrulewidth\vskip\footruleskip}}
\renewcommand{\headrulewidth}{1.5pt}
\renewcommand{\footrulewidth}{1.5pt}

\usepackage{caption}
\usepackage{subcaption}

\spanishdecimal{.}

\begin{document}
\fancyhead{}
\fancyfoot{}
\fancyhead[L]{
\begin{minipage}{3.5cm}
\begin{center}
	\includegraphics[width=0.95\textwidth]{logousb.png}
\end{center}
\end{minipage}
\begin{minipage}{12cm}
\begin{flushleft}
\small \textsc{Universidad de San Buenaventura}\\
\small \textsc{School of Engineering}\\
\small \textsc{Mechatronics Engineering\\}
\end{flushleft}
\end{minipage}
}
\fancyhead[R]{
\begin{minipage}{3.0cm}
\begin{flushright}
\small \textsc{Mechanics of Materials \\ 3rd Term}\\
\small \textsc{2021-I}
\end{flushright}
\end{minipage}
}
\fancyfoot[R]{\large \textbf{\thepage}}

\begin{minipage}{0.3\textwidth}
\begin{flushleft}
\textbf{Author:}\\
\textit{Nikolay Prieto Ph.D(c)}\\
\end{flushleft}
\end{minipage}
\begin{minipage}{0.7cm}
\textcolor{gray}{\rule{0.3cm}{2.5cm}}
\end{minipage}
\begin{minipage}{0.64\textwidth}
\Large{\textbf{Computational Laboratory \\ ANSYS (Part II \\ Elelements and Meshes)}}
\end{minipage}\\

\noindent
\textcolor{gray}{\rule{\textwidth}{0.5pt}}\\
\renewcommand{\tablename}{Tabla}
\renewcommand{\arraystretch}{1.2}
\renewcommand\contentsname{Outline}
\tableofcontents

\noindent
\textcolor{gray}{\rule{\textwidth}{0.5pt}}\\

\section{Introduction}

In the previous laboratory, we discussed some of the assumptions that can be used to simplify and reduce the dimensionality of a Finite Element Model (FEM). For example, if a beam has a constant cross section, it can be modeled as a one-dimensional body. If a beam is symmetric with a uniform cross section, it can be modeled as a two-dimensional body. Or, it can be modeled as three dimensional body. If created correctly, all three types of models will produce the same results. However, they will have different solid model geometry and must be meshed using different element types (Fig. \ref{fig:cantilever} )

\begin{figure}[h]
   \centering
   \includegraphics[width=0.6\textwidth]{cantileverbeam}
   \caption{Element Plots of a Square Cantilever Beam Modeled using 1D Beam Elements (BEAM189, top), 2D Continuum Elements (PLANE183, center), and 3D Continuum Elements (SOLID185, bottom).}
   \label{fig:cantilever}
\end{figure}

\section{Elements}

The Ansys library element is available to be used dependent on the purpose. There are specific elements for heat transfer, static structural, fluid dynamics or transient (time domain) problems. Each ANSYS element has a number of properties including its name, characteristic and degen- erate shapes, number of nodes, degrees of freedom, real constants, key options, material properties, permitted loads, and special features. An overview of these features can be found in the input summary of each element in the ANSYS Element Library. The input summary follows the element description and the element input data.

Each element in ANSYS has a name followed by a number. The combination of a name and a number cannot exceed 8 characters. The name indicates the element’s family (FLUID, PLANE, SHELL, SOLID, etc.). The number is a unique identifier called the element routine number. For example, a PLANE182 element is a member of the PLANE element family and it is element routine number 182. When element types are defined using the command line or input files, only the routine number is required. The element family name is optional.

\subsection{Element shapes}

There are eight possible element shapes in ANSYS: points (for point elements); lines (for line elements); triangles or quadrilaterals (for area elements); and tetrahedrons, pyramids, prisms, or bricks (for volume elements). These shapes are shown in Fig. \ref{fig:shapes}

\begin{figure}[h]
   \centering
   \includegraphics[width=0.95\textwidth]{elementshapes}
   \caption{Possible Element Shapes in ANSYS.}
   \label{fig:shapes}
\end{figure}

\subsection{Number of nodes}

All elements have nodes that define their location in space. For example, quadrilateral elements like PLANE55 have four nodes (I, J, K, and L)—one for each corner (Fig. \ref{fig:nodes}, left). For example, PLANE77 is the counterpart to PLANE55 with midside nodes. It has a total of eight nodes (I, J, K, L, M, N, O and P) - one for each corner and one in the middle of each side (Figure \ref{fig:nodes}, right). The node letters refer to the position of the nodes on a generic element (i.e., on the element type). The node lettering convention starts with the letter I and increments one letter per node.

\begin{figure}[h]
   \centering
   \includegraphics[width=0.6\textwidth]{nodesinelement}
   \caption{Quadrilateral Elements Without and With Mid-Side Nodes.}
   \label{fig:nodes}
\end{figure}

Quadrilateral and brick elements can be forced into degenerate shapes with one or more triangular faces in order to mesh geometries that could not be meshed otherwise. For example, a quadrilateral element may collapse into a triangle (Figure 4.7, \ref{fig:nodes_elements}), while a brick element may collapse into a prism, a pyramid, or a tetrahedron. This is achieved by defining the same node number for multiple nodes (Figure \ref{fig:nodes_elements}, right).

\begin{figure}[h]
   \centering
   \includegraphics[width=0.9\textwidth]{shapesandnodes}
   \caption{Quadrilateral Elements with Default and Degenerate Shapes: Node Letters for a Generic Element (left) and Node Numbers for a Specific Element in a FE Mesh (right)}
   \label{fig:nodes_elements}
\end{figure}

\end{document}